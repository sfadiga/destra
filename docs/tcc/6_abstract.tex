\pagestyle{empty}

\chapter*{Abstract}

\noindent
FADIGA, S. F. M. \textbf{Debugging Real-Time Systems}: A Code Instrumentation Approach for
Testing Critical Systems. 2025. Monografia (Trabalho de Conclusão de Curso) –
Escola de Engenharia de São Carlos, Universidade de São Paulo, São Carlos, 2025.

\vspace*{5mm}

\noindent
The development of real-time embedded systems presents significant challenges with respect to requirements verification and validation, especially in integrated test scenarios where access to internal software states is limited. Traditional debugging tools, such as probes and JTAG interfaces, are widely used during early development phases; however, they present limitations when applied to integration testing or during normal system execution.
In this context, this work proposes, implements, and evaluates a simple \textit{peek} and \textit{poke} protocol aimed at the controlled observation and modification of internal states in real-time embedded systems, in order to support testing and debugging activities. The proposed protocol enables the reading and writing of internal variables during software execution, without relying on intrusive or proprietary debugging interfaces.
As a proof of concept, the protocol was implemented on a low-cost microcontroller and integrated with a host-side client tool, allowing interaction with the embedded system through deterministic commands. The solution was evaluated in representative test scenarios, demonstrating its applicability as an auxiliary debugging mechanism in integrated testing environments.
The results indicate that the proposed approach can contribute to the inspection of internal software states during execution while preserving the system’s temporal behavior, thus supporting requirements validation in embedded systems.

\vspace*{5mm}

\noindent
\textbf{Keywords:} real-time systems; integration testing; peek/poke.
