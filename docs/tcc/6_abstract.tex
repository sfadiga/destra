\pagestyle{empty}

\chapter*{Abstract}

\noindent
FADIGA, S. F. M. \textbf{Debugging Real-Time Systems}: A Code Instrumentation Approach for
Testing Critical Systems. 2025. Monografia (Trabalho de Conclusão de Curso) –
Escola de Engenharia de São Carlos, Universidade de São Paulo, São Carlos, 2025.

\vspace*{5mm}

\noindent
The development of real-time embedded systems presents significant challenges regarding the verification and validation of requirements, especially in integrated testing scenarios, where access to internal software states is limited. Traditional debugging tools, such as probes and JTAG interfaces, are widely used in the early development phases, but they present restrictions when applied to integration testing or normal system operation.
In this context, this work proposes, implements, and evaluates a simple \textit{peek} and \textit{poke} protocol aimed at the controlled observation and modification of internal states of real-time embedded systems, with the objective of supporting debugging, testing, and verification activities. The protocol enables reading and writing of internal variables during software execution, without the need for proprietary or intrusive debugging interfaces.
As a proof of concept, the protocol was implemented on a low-cost microcontroller and integrated with a client tool on the host, enabling interaction with the embedded system through deterministic commands. The operation of the solution was evaluated in representative test scenarios, demonstrating its applicability as an auxiliary debugging mechanism in integrated testing environments.
The results indicate that the proposed approach can contribute to the inspection of internal software states during execution, while preserving the system's temporal behavior and providing support for requirements verification in embedded systems.

\vspace*{5mm}

\noindent
\textbf{Keywords:} real-time systems; integration testing; \textit{peek} and \textit{poke}.
