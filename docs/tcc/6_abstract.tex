\pagestyle{empty}

\chapter*{Abstract}

\noindent
FADIGA, S. F. M. \textbf{Debugging Real-Time Systems}: A Code Instrumentation Approach for
Testing Critical Systems. 2025. Monografia (Trabalho de Conclusão de Curso) –
Escola de Engenharia de São Carlos, Universidade de São Paulo, São Carlos, 2025.

\vspace*{5mm}

\noindent
The development of real-time embedded systems presents specific challenges regarding requirement verification and validation, especially in integrated testing environments where debugging capabilities are limited. Traditional debugging tools, such as probes and JTAG interfaces, cannot be used during normal software execution, requiring alternative approaches. This work proposes the development of an embedded code instrumentation tool, focused on monitoring and injecting values into variables during system execution. The solution comprises three main modules: a simple communication protocol with embedded code, a C language library for command interpretation and internal variable access, and a graphical host interface for interaction with the embedded system. The tool is designed for integrated testing, such as in Hardware-in-the-Loop (HIL) environments, and aims to assist engineers in validating critical systems. As a final product, a proof of concept (PoC) will be delivered demonstrating the tool's functionality, implementing real-time variable monitoring (Peek) and modification (Poke) operations.

\vspace*{5mm}

\noindent
\textbf{Keywords:} Embedded Systems, Code Instrumentation, Value Injection, Variable Monitoring, Integrated Testing, Software Verification, Critical Software, Real-Time Software.
