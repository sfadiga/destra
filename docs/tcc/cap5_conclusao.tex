\chapter{Conclusão e Trabalhos Futuros}

\label{chap:conclusao}

\section{Considerações Finais}

A depuração e a verificação de sistemas embarcados críticos de tempo real representam esafios contínuos, especialmente quando há a necessidade de comprovar requisitos funcionais e temporais sem interferir diretamente na execução do software. Este trabalho apresentou o desenvolvimento e validação de um protocolo simples de peek/poke, voltado para observação e modificação controlada de variáveis internas, aplicado a um microcontrolador de baixo custo e com foco em apoio às atividades de teste e certificação conforme os princípios estabelecidos pela DO-178C.

A proposta demonstrou que é possível projetar uma ferramenta de depuração modular, rastreável e de baixo custo, com impacto mínimo sobre o comportamento temporal do sistema embarcado. O uso de comandos estruturados e medições internas de desempenho (via \texttt{CMD\_GET\_PERF\_LOG}) permitiu validar a integridade da execução e quantificar a sobrecarga introduzida pela comunicação, confirmando a viabilidade da solução como mecanismo auxiliar em campanhas de teste e integração de software crítico.

\section{Conclusões sobre o Protocolo Desenvolvido}

Os resultados obtidos nos testes de desempenho e estresse mostraram que o protocolo mantém latência média inferior a 10 ms e jitter reduzido, mesmo sob alta carga de comandos, o que evidencia sua estabilidade temporal e robustez de comunicação. A análise embarcada revelou ainda que o tempo de processamento de comandos (\texttt{command\_process\_time}) manteve-se estável, em torno de 0,73 ms, com desvio padrão inferior a 0,003 ms, confirmando a previsibilidade exigida para sistemas de tempo real.

Além de cumprir seu objetivo principal — prover leitura e escrita em variáveis internas de forma segura e controlada —, o protocolo mostrou-se útil como instrumento de coleta de dados para verificação de requisitos temporais, permitindo correlacionar métricas internas (como frame rate e process time) com medições externas via osciloscópio. Essa correlação fornece uma camada adicional de rastreabilidade entre o comportamento observado e o comportamento esperado, o que é fundamental em atividades de verificação e validação sob normas de certificação.

\section{Relevância para o Contexto de Certificação}

No contexto da certificação de software embarcado, a rastreabilidade entre requisitos, código e evidências de teste é um elemento essencial. O protocolo desenvolvido contribui diretamente para esse processo ao permitir a observação controlada de estados internos sem necessidade de instrumentação invasiva nem dependência de ferramentas proprietárias.

Conforme previsto nas normas \cite{rtca2011do178c} e \cite{rtca2012do330}, a qualificação de ferramentas de apoio depende de seu impacto sobre as atividades de verificação. Nesse sentido, a solução proposta se enquadra como uma ferramenta de suporte não intrusiva, podendo ser utilizada em fases de integração e ensaio funcional para coleta de evidências objetivas de execução de requisitos temporais, análise de desempenho e confirmação de margens de WCET (\textit{Worst Case Execution Time}).

A principal contribuição está em demonstrar que é possível implementar, de forma independente e rastreável, uma ferramenta simples, de arquitetura aberta e documentação completa, capaz de apoiar testes de conformidade e verificação de requisitos em sistemas embarcados críticos, mesmo quando não se dispõe de ferramentas comerciais qualificadas. Isso torna o framework aplicável tanto em ambientes acadêmicos (para ensino de práticas de verificação embarcada) quanto em contextos industriais, como protótipo de infraestrutura de depuração aderente aos processos de certificação.

\section{Limitações Observadas}

Durante a execução dos testes e da validação funcional do protocolo, algumas limitações foram identificadas e são consideradas oportunidades de evolução.

\subsection{Ausência de Operações Contínuas}

A primeira limitação diz respeito à ausência de suporte a operações contínuas de leitura ou escrita (\textit{continuous peek e poke}). Atualmente, cada requisição precisa ser iniciada de forma individual, o que limita o uso do protocolo em medições de alta frequência ou em observações contínuas de variáveis dinâmicas. A inclusão de um modo de operação contínua permitiria capturar séries temporais de forma mais eficiente, reduzindo a sobrecarga de comandos e ampliando o potencial de análise temporal.

\subsection{Falta de Mecanismos de Integridade}

Outra limitação está relacionada à robustez da camada de verificação de integridade de dados. Embora a comunicação se mostre estável nas condições atuais, o protocolo ainda não implementa mecanismos de verificação de redundância cíclica (CRC) ou checagem de erro robusta nos pacotes trocados. A ausência de um CRC dedicado pode, em cenários de ruído eletromagnético ou baud rates mais elevadas, introduzir risco de corrupção silenciosa de dados. A adoção de um CRC-16 ou CRC-32, associado à confirmação de sequência de comandos, representaria um avanço importante na confiabilidade e na segurança da comunicação.

\subsection{Limitações de Arquitetura}

Por fim, destaca-se que o framework, em sua forma atual, foi projetado para ambientes monothread e de comunicação serial simples, o que limita seu uso direto em plataformas com sistemas operacionais de tempo real (RTOS) ou múltiplas tarefas concorrentes. Apesar disso, sua estrutura modular permite fácil extensão para suportar filas de mensagens, buffers circulares e mecanismos de sincronização adequados para tais contextos.

\section{Trabalhos Futuros}

Dando continuidade ao trabalho, propõem-se as seguintes evoluções e aprimoramentos:

\begin{enumerate}
    \item \textbf{Implementação de Operações Contínuas:} Suporte a \textit{continuous peek e poke}, permitindo leitura e escrita contínua em variáveis selecionadas com temporização configurável.
    \item \textbf{Mecanismos de Integridade de Dados:} Inclusão de verificação de redundância cíclica (CRC) nos pacotes de comunicação, garantindo detecção de erros de transmissão e validação completa dos dados trocados.
    \item \textbf{Suporte a RTOS e Multitarefa:} Extensão do protocolo para ambientes com sistemas operacionais de tempo real, de forma a permitir sua aplicação em sistemas mais complexos e próximos de aplicações aeronáuticas reais.
    \item \textbf{Integração com Ferramentas de Teste:} Integração com ferramentas de teste automatizado e bancos de requisitos, permitindo o registro automático de medições como evidências de verificação.
    \item \textbf{Interface Gráfica Avançada:} Desenvolvimento de interface gráfica (GUI) aprimorada para facilitar a configuração, monitoramento e registro de sessões de depuração, com suporte a exportação de relatórios e visualização de históricos.
    \item \textbf{Qualificação conforme DO-330:} Estudo de qualificação da ferramenta conforme os critérios do DO-330 \cite{rtca2012do330}, avaliando seu enquadramento como ferramenta de apoio à verificação.
    \item \textbf{Suporte a Interfaces de Alta Velocidade:} Adaptação do protocolo para interfaces mais rápidas (CAN, SPI, Ethernet), de modo a ampliar o alcance e a taxa de atualização para cenários de teste em tempo real.
    \item \textbf{Validação em Plataformas Adicionais:} Portabilidade do protocolo para outras arquiteturas de microcontroladores (ARM Cortex-M, RISC-V) e sistemas operacionais embarcados.
    \item \textbf{Documentação e Padronização:} Elaboração de especificação formal do protocolo e publicação de guidelines para implementação em diferentes plataformas, facilitando adoção e contribuições da comunidade.
\end{enumerate}

\section{Perspectivas Finais}

As melhorias propostas fortaleceriam ainda mais o protocolo como instrumento de apoio aos testes de certificação, tornando-o um componente efetivo na geração de evidências rastreáveis e no monitoramento controlado de software crítico em execução.

A consolidação dessas evoluções permitiria ao framework DESTRA ser adotado não apenas como ferramenta de prototipagem e desenvolvimento, mas também como componente qualificado em processos formais de verificação e validação de sistemas embarcados críticos.

Espera-se que este trabalho estimule futuras pesquisas e desenvolvimentos na área de depuração e monitoramento de sistemas embarcados, contribuindo para uma indústria mais segura, rastreável e aderente aos mais rigorosos padrões de certificação internacional.

\section{Resumo das Contribuições}

As principais contribuições deste trabalho podem ser assim resumidas:

\begin{itemize}
    \item Desenvolvimento de um protocolo simples, determinístico e modular para operações \textit{peek} e \textit{poke} em sistemas embarcados de baixo custo.
    \item Implementação completa do protocolo em Arduino UNO com instrumentação para coleta de métricas de desempenho.
    \item Desenvolvimento de ferramenta host em Python com interface gráfica intuitiva, suportando carregamento de símbolos ELF/DWARF e operações de monitoramento em tempo real.
    \item Validação experimental abrangente através de testes de latência, estresse e rajada, com medições correlacionadas por osciloscópio.
    \item Demonstração de viabilidade da solução para contextos de certificação de software crítico de segurança, conforme normas DO-178C e DO-330.
    \item Disponibilização de solução de código aberto e documentação completa para apoio a pesquisa e desenvolvimento acadêmico e industrial.
\end{itemize}
