% ============================================================================
% APÊNDICE E - DIAGRAMA DE CLASSES DESTRA UI
% ============================================================================
\chapter{Diagrama de Classes da Aplicação DESTRA UI}

Este apêndice apresenta o diagrama de classes UML da aplicação DESTRA UI, mostrando a arquitetura do sistema e os relacionamentos entre os principais componentes.

\section{Visão Geral da Arquitetura}

A aplicação DESTRA UI foi desenvolvida seguindo uma arquitetura em camadas, separando claramente as responsabilidades:

\begin{itemize}
    \item \textbf{Camada de Interface}: Responsável pela interação com o usuário (DestraGUI)
    \item \textbf{Camada de Protocolo}: Gerencia a comunicação serial com o Arduino (DestraProtocol)
    \item \textbf{Camada de Análise}: Processa arquivos ELF e extrai informações de variáveis (ElfDataDictionary)
    \item \textbf{Camada de Dados}: Define estruturas para armazenar informações (VariableInfo, PerformanceData)
\end{itemize}

\section{Diagrama de Classes}

\begin{figure}[H]
\centering
\begin{tikzpicture}[scale=1.0, transform shape]

% DestraGUI (Interface Principal)
\begin{class}[text width=4cm]{DestraGUI}{3,6}
\attribute{- logger\_manager: DestraLogger}
\attribute{- elf\_data: ElfDataDictionary}
\attribute{- \_destra: DestraProtocol}
\attribute{- all\_variables: List[VariableInfo]}
\attribute{- auto\_peek\_timer: QTimer}
\attribute{- \_is\_connected: bool}
\operation{+ connect\_to\_arduino(): void}
\operation{+ browse\_file(): void}
\operation{+ peek\_values(): void}
\operation{+ poke\_values(): void}
\operation{+ start\_stop\_auto\_peek(): void}
\end{class}

% DestraProtocol (Protocolo de Comunicação)
\begin{class}[text width=4cm]{DestraProtocol}{8,6}
\attribute{- MAGIC\_CA: bytes}
\attribute{- MAGIC\_FE: bytes}
\attribute{- PEEK\_CMD: bytes}
\attribute{- POKE\_CMD: bytes}
\attribute{- ser: serial.Serial}
\attribute{- logger: Logger}
\operation{+ connect(): bool}
\operation{+ peek(address, size): bytes}
\operation{+ poke(address, value): bool}
\operation{+ performance(): List}
\end{class}

% ElfDataDictionary (Análise de Arquivos ELF)
\begin{class}[text width=4.5cm]{ElfDataDictionary}{-3,6}
\attribute{- elf\_file: ELFFile}
\attribute{- dwarf\_info: DWARFInfo}
\attribute{- variables: Dict}
\attribute{- logger: Logger}
\operation{+ load\_elf\_file(path): bool}
\operation{+ get\_variables(): List}
\operation{+ search\_variables(pattern): List}
\end{class}

% VariableInfo (Informações de Variável)
\begin{class}[text width=3cm]{VariableInfo}{-2,-3}
\attribute{+ name: str}
\attribute{+ address: int}
\attribute{+ size: int}
\attribute{+ base\_type: str}
\attribute{+ is\_signed: bool}
\operation{+ \_\_str\_\_(): str}
\end{class}

% Relacionamentos
\draw[umlcd style dashed line] (DestraGUI) -- node[above, sloped, black]{usa} (DestraProtocol);
\draw[umlcd style dashed line] (DestraGUI) -- node[left, black]{usa} (ElfDataDictionary);
\draw[umlcd style dashed line] (ElfDataDictionary) -- node[above, sloped, black]{cria} (VariableInfo);

\end{tikzpicture}
\caption{Diagrama de Classes da Aplicação DESTRA UI}
\label{fig:diagrama_classes_destra}
\end{figure}

\section{Descrição das Classes}

\subsection{DestraGUI}
Classe principal da interface gráfica, herda de \texttt{QMainWindow}. Responsável por:
\begin{itemize}
    \item Gerenciar a interface do usuário com PySide6/Qt
    \item Coordenar as operações entre protocolo e análise ELF
    \item Implementar funcionalidades de auto-peek
    \item Gerenciar tabelas de variáveis disponíveis e selecionadas
\end{itemize}

\subsection{DestraProtocol}
Implementa o protocolo de comunicação serial com o Arduino:
\begin{itemize}
    \item Define comandos PEEK, POKE e PERFORMANCE
    \item Gerencia conexão serial e detecção automática de portas
    \item Processa respostas do protocolo com verificação de integridade
    \item Decodifica tipos de dados recebidos
\end{itemize}

\subsection{ElfDataDictionary}
Analisador de arquivos ELF com informações DWARF:
\begin{itemize}
    \item Extrai informações de variáveis (nome, endereço, tipo, tamanho)
    \item Processa estruturas, arrays e tipos básicos
    \item Fornece funcionalidades de busca e filtragem
    \item Gera estatísticas do arquivo analisado
\end{itemize}

\subsection{VariableInfo}
Estrutura de dados (dataclass) que armazena informações de uma variável:
\begin{itemize}
    \item Nome da variável
    \item Endereço de memória
    \item Tamanho em bytes
    \item Tipo base (uint8, uint16, uint32, etc.)
    \item Informações sobre sinalização e ponteiros
\end{itemize}

\subsection{DecodedTypes}
Classe utilitária para decodificação de tipos de dados:
\begin{itemize}
    \item Mapeia tipos de dados para formatos struct
    \item Fornece informações de tamanho por tipo
    \item Lista tipos suportados pelo sistema
\end{itemize}

\section{Padrões de Design Utilizados}

\subsection{Singleton}
A classe \texttt{DestraLogger} implementa o padrão Singleton para garantir uma instância única de configuração de logging em toda a aplicação.

\subsection{Model-View-Controller (MVC)}
A arquitetura segue parcialmente o padrão MVC:
\begin{itemize}
    \item \textbf{Model}: \texttt{ElfDataDictionary}, \texttt{VariableInfo}, \texttt{PerformanceData}
    \item \textbf{View}: \texttt{DestraGUI} e componentes Qt (QTableWidget, QComboBox, etc.)
    \item \textbf{Controller}: Métodos da \texttt{DestraGUI} que coordenam Model e View
\end{itemize}

\subsection{Data Transfer Object (DTO)}
As classes \texttt{VariableInfo} e \texttt{PerformanceData} funcionam como DTOs, transportando dados estruturados entre diferentes camadas da aplicação.

\subsection{Facade}
A classe \texttt{DestraProtocol} atua como uma facade, simplificando o acesso ao protocolo de comunicação serial complexo.
