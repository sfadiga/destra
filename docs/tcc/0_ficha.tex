\newpage
% Texto de autorização no topo
{\small
\noindent
AUTORIZO A REPRODUÇÃO TOTAL OU PARCIAL DESTE TRABALHO, \\
POR QUALQUER MEIO CONVENCIONAL OU ELETRÔNICO, PARA FINS \\
DE ESTUDO E PESQUISA, DESDE QUE CITADA A FONTE.
}

\vspace{3cm}

% Linha da instituição
{\small
\noindent
Ficha catalográfica elaborada pela Biblioteca Prof. Dr. Sérgio Rodrigues Fontes da \\
EESC/USP com os dados inseridos pelo(a) autor(a).
}

\vspace{0.5cm}

% Caixa da ficha
\noindent
\begin{tabular}{|p{1.2cm}|p{13.5cm}|}
\hline
\multirow{16}{1.2cm}{F145d} & 
\textbf{Fadiga, Sandro Ferraz Martins} \\
& \quad Depuração de Sistemas em Tempo Real: Uma Abordagem de \\
& \quad Instrumentação de Código para Testes de Sistemas \\
& \quad Críticos.  / Sandro Ferraz Martins Fadiga; orientador \\
& \quad Glauco Augusto de Paula Caurin. São Carlos, 2025. \\
& \\
& \quad Especialização (Especialização em Sistemas \\
& \quad aeronáuticos) -- Escola de Engenharia de São Carlos da \\
& \quad Universidade de São Paulo, 2025. \\
& \\
& \quad 1. Sistemas embarcados. 2. Instrumentação de \\
& \quad código. 3. Injeção de valores. 4. Monitoramento de \\
& \quad variáveis. 5. Testes integrados. 6. Verificação de \\
& \quad software. 7. Software crítico. 8. Software tempo real. \\
& \quad I. Título. \\
\hline
\end{tabular}

\vspace{1cm}

% Rodapé
{\small
\noindent
Eduardo Graziosi Silva - CRB - 8/8907
}
