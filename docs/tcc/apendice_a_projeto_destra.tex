% ============================================================================
% APÊNDICE A - PROJETO DESTRA NO GITHUB
% ============================================================================
\appendix
\chapter{Repositório do Projeto DESTRA}
\label{apx:repositorio_destra}

Devido à extensão do código-fonte desenvolvido ao longo deste trabalho — incluindo o firmware embarcado, a ferramenta host, os módulos de instrumentação e os scripts de teste — optou-se por não incluir o código completo como apêndice impresso neste documento.

A disponibilização integral de aproximadamente sessenta páginas de código comprometeria a legibilidade do trabalho e não agregaria valor técnico adicional à análise apresentada nos capítulos anteriores. Em consonância com práticas modernas de engenharia de software e pesquisa experimental, todo o código-fonte foi organizado e disponibilizado em um repositório público versionado.

O repositório oficial do projeto DESTRA está disponível em:

\begin{quote}
  \url{https://github.com/sfadiga/destra/}
\end{quote}

O repositório contém todos os artefatos necessários para reprodução, validação e extensão da solução proposta, incluindo o histórico completo de desenvolvimento.

\subsection{Estrutura do Repositório}

A organização do repositório segue uma separação clara de responsabilidades entre os componentes do sistema, conforme descrito a seguir:

\begin{itemize}
    \item \texttt{/arduino/} — Código-fonte do firmware embarcado responsável pela implementação do protocolo DESTRA no microcontrolador (desenvolvido em linguagem C do Arduino).
    Inclui:
    \begin{itemize}
        \item \texttt{/destra\_protocol\_test/} — Firmware utilizado nos ensaios experimentais e testes de desempenho.
        Inclui:
        \begin{itemize}
          \item \texttt{/arduino/destra\_protocol\_test/destra\_protocol\_test.ino} — Implementação do protocolo para Arduino instrumentado para testes.
          \item \texttt{/arduino/destra\_protocol\_test/sample.ino} — Exemplo de integração.
        \end{itemize}
        \item \texttt{/sample/} — Firmware mínimo de exemplo para demonstração e validação funcional do protocolo.
        Inclui:
        \begin{itemize}
          \item \texttt{/arduino/sample/destra\_protocol.ino} — Implementação do protocolo para Arduino.
          \item \texttt{/arduino/sample/sample.ino} — Exemplo de integração.
        \end{itemize}
    \end{itemize}

    \item \texttt{/src/} — Código-fonte da aplicação \textit{host}, desenvolvida em Python, responsável pela comunicação com o dispositivo embarcado, coleta de dados, instrumentação e interface gráfica utilizada nos experimentos.
    Inclui:
    \begin{itemize}
        \item \texttt{/src/data\_dictionary.py} — Parser de arquivos ELF/DWARF.
        \item \texttt{/src/destra\_ui.py} — Interface gráfica (PySide/Qt).
        \item \texttt{/src/destra.py} — Implementação do protocolo no host.
        \item \texttt{/src/logger\_config.py} — Sistema de logging.
        \item \texttt{/src/requirements.txt} — Dependências Python.
      \end{itemize}
    \item \texttt{/tests/} — Scripts e rotinas de teste empregados na execução dos ensaios descritos neste trabalho, incluindo testes de latência, rajada e análise estatística. Dados legados de testes preliminares são mantidos apenas para referência histórica.
    \item \texttt{/logs/} — Arquivos de registro gerados durante a execução dos testes, contendo medições brutas de tempo, eventos de comunicação e informações de depuração, utilizados posteriormente na análise experimental.
    \item \texttt{/docs/} — Documentação complementar do projeto, incluindo os arquivos \LaTeX{} deste trabalho acadêmico, figuras, dados experimentais consolidados e materiais de apoio à banca examinadora.
\end{itemize}

\subsection{Reprodutibilidade}

O repositório contém o arquivo \texttt{README.md} localizado no diretório raiz do projeto, este inclui instruções detalhadas para: compilação do firmware, execução da ferramenta host e realização dos testes experimentais apresentados neste trabalho, permitindo a completa reprodutibilidade dos resultados obtidos.


