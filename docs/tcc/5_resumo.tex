\newpage

\pagestyle{empty}

\chapter*{Resumo}

\noindent
FADIGA, S. F. M. \textbf{Depuração de Sistemas em Tempo Real}: Uma Abordagem de Instrumentação de Código para Testes de Sistemas Críticos. 2025. Monografia (Trabalho de Conclusão de Curso) – Escola de Engenharia de São Carlos, Universidade de São Paulo, São Carlos, 2025.

\vspace*{5mm}

\noindent
O desenvolvimento de sistemas embarcados de tempo real apresenta desafios específicos no que se refere à verificação e validação de requisitos, especialmente em ambientes de testes integrados, onde a capacidade de depuração é limitada. Ferramentas tradicionais de depuração, como probes e interfaces JTAG, não podem ser utilizadas durante a execução normal do software, exigindo abordagens alternativas. Este trabalho propõe o desenvolvimento de uma ferramenta de instrumentação de software embarcado, focada no monitoramento e injeção de valores em variáveis durante a execução dos sistemas. A solução será composta por três módulos principais: um protocolo simples de comunicação com o software embarcado, uma biblioteca em linguagem C para interpretação de comandos e acesso a variáveis internas, e uma interface gráfica no host para interação com o sistema embarcado. A ferramenta será projetada para testes integrados, como em ambientes Hardware-in-the-Loop (HIL), e visa auxiliar engenheiros na validação de sistemas críticos. Como produto final, será entregue uma prova de conceito (PoC) demonstrando o funcionamento da ferramenta, implementando operações de monitoramento (peek) e modificação (poke) de variáveis em tempo real.

\vspace*{5mm}

\noindent
\textbf{Palavras-chave:} Sistemas Embarcados, Instrumentação de Código, Injeção de valores, Monitoramento de Variáveis, Testes Integrados, Verificação de software, Software crítico, Software de tempo real.
