\newpage

\pagestyle{empty}

\chapter*{Resumo}

\noindent
FADIGA, S. F. M. \textbf{Depuração de Sistemas em Tempo Real}: Uma Abordagem de Instrumentação de Código para Testes de Sistemas Críticos. 2025. Monografia (Trabalho de Conclusão de Curso) – Escola de Engenharia de São Carlos, Universidade de São Paulo, São Carlos, 2025.

\vspace*{5mm}

\noindent
O desenvolvimento de sistemas embarcados de tempo real apresenta desafios significativos no que se refere à verificação e validação de requisitos, especialmente em cenários de teste integrados, nos quais o acesso a estados internos do software é limitado. Ferramentas tradicionais de depuração, como sondas e interfaces JTAG, são amplamente utilizadas nas fases iniciais de desenvolvimento, porém apresentam restrições quando aplicadas a testes de integração ou à execução normal do sistema.
Neste contexto, este trabalho propõe, implementa e avalia um protocolo simples de \textit{peek} e \textit{poke} voltado à observação e modificação controlada de estados internos de sistemas embarcados de tempo real, com o objetivo de apoiar atividades de depuração, teste e verificação. O protocolo permite a leitura e a escrita de variáveis internas durante a execução do software, sem a necessidade de interfaces de depuração proprietárias ou intrusivas.
Como prova de conceito, o protocolo foi implementado em um microcontrolador de baixo custo e integrado a uma ferramenta cliente no host, possibilitando a interação com o sistema embarcado por meio de comandos determinísticos. O funcionamento da solução foi avaliado em cenários de teste representativos, demonstrando sua aplicabilidade como mecanismo auxiliar de depuração em ambientes de testes integrados.
Os resultados indicam que a abordagem proposta pode contribuir para a inspeção de estados internos do software durante a execução, preservando o comportamento temporal do sistema e oferecendo suporte à verificação de requisitos em sistemas embarcados.

\vspace*{5mm}

\noindent
\textbf{Palavras-chave:} sistemas de tempo-real; testes integrados; \textit{peek} e \textit{poke}.
