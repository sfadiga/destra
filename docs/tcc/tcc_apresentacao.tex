\documentclass[10pt,aspectratio=169]{beamer} % 16:9, mais moderno


% Tema recomendado
\usetheme{Metropolis}
\usecolortheme{default}

% Pacotes básicos
\usepackage[utf8]{inputenc}
\usepackage[brazil]{babel}
\usepackage{graphicx}
\usepackage{hyperref}
\usepackage{booktabs}


% Configurações pessoais
\title{Depuração de Sistemas em Tempo Real:}
\subtitle{Uma Abordagem de Instrumentação de Código para Testes de Sistemas Críticos}
\author{Sandro Fadiga\\[0.2cm]Orientador: Prof. Dr. Glauco Augusto de Paula Caurin}
\date{2025}
\institute{Escola de Engenharia de São Carlos Universidade de São Paulo - USP}

% Cores personalizadas (opcional)
%\definecolor{azuluno}{RGB}{0,51,102}
%\setbeamercolor{title}{fg=azuluno}
%\setbeamercolor{frametitle}{fg=azuluno}
%\setbeamercolor{structure}{fg=azuluno}

\begin{document}

% Slide 1 - Capa
\begin{frame}
    \titlepage
\end{frame}

% Slide 2 - Sumário da apresentação
%\begin{frame}{Sumário da Apresentação}
%    \tableofcontents
%\end{frame}

\section{1. Introdução}
\begin{frame}{Introdução}
    \begin{itemize}
        \item Contextualização
        \item Objetivos
        \item Justificativa
        \item Desenvolvimento
        \item Testes e Resultados
        \item Conclusão
    \end{itemize}
\end{frame}

\section{2. Fundamentação Teórica}
\begin{frame}{2.1 Contextualização}
\begin{figure}[htbp]
    \centering
    \includegraphics[height=0.8\textheight]{recursos/figuras/iron_bird.png}
    \caption{Foto de um típico ambiente de ensaios integrados (Iron Bird).}
\end{figure}    
\end{frame}

\begin{frame}{2.2 Objetivos}
    \begin{block}{Objetivo Geral}
        \vspace{0.3cm}
        Desenvolver e validar um protocolo leve de depuração (peek/poke/telemetria) para sistemas embarcados críticos com mínima intrusão temporal.
    \end{block}
    \begin{block}{Objetivos Específicos}
        \begin{itemize}
            \item Especificar protocolo binário eficiente
            \item Implementar no microcontrolador e host
            \item Avaliar impacto temporal em diferentes taxas
        \end{itemize}
    \end{block}
\end{frame}

\begin{frame}{2.3 Justificativa}
    \begin{columns}
        \column{0.5\textwidth}
            \textbf{Tópicos abordados:}
            \begin{itemize}
                \item Software embarcado crítico
                \item Tempo real e determinismo
                \item Setores de alta criticidade
                \item Testes e certificação (DO-178C, ISO 26262, IEC 61508/62304)
            \end{itemize}
        \column{0.5\textwidth}
    \end{columns}
\end{frame}

\section{3 Desenvolvimento}
\begin{frame}{3.1 Protocolo}
    \begin{itemize}
        \item Arquitetura do Sistema
        \item Especificação do protocolo
        \item Máquina de Estados
        \item Cliente Host – Desenvolvimento
        \item Considerações de extensibilidade
    \end{itemize}
\end{frame}

\begin{frame}{3.2 Arquitetura do Sistema}
    \centering
    \includegraphics[width=0.9\textwidth]{recursos/figuras/arquitetura_sistema.png}
\end{frame}

\begin{frame}[fragile]{3.2 Especificação do protocolo}
    Examplo do Comando Peek:
    \begin{center}
        \texttt{CA FE F1 04 01 02}
    \end{center}
    Decomposição:
    \begin{itemize}
        \item \texttt{CA FE}: Palavras mágicas
        \item \texttt{F1}: Comando PEEK
        \item \texttt{04}: Byte baixo do endereço (LSB)
        \item \texttt{01}: Byte alto do endereço (MSB)
        \item \texttt{02}: Tamanho (2 bytes)
    \end{itemize}
    \vspace{0.5cm}
    Nota: o endereço \texttt{0x0104} é transmitido como \texttt{04 01} (little-endian).
\end{frame}

\begin{frame}{3.3 Máquina de Estados}
    \centering
    \includegraphics[width=0.65\textheight]{recursos/figuras/destra_diagrama_estados.png}
\end{frame}

\begin{frame}{3.4 Cliente Host – Desenvolvimento}
    \begin{itemize}
        \item Ferramenta escrita em Python/Qt
        \item Usa \texttt{pyelftools} ou \texttt{libdwarf}
        \item Resolve símbolos automaticamente a partir do arquivo elf/dwarf
        \item Comandos simples: \texttt{peek 'variavel'}, \texttt{poke 'variavel' = 42}
    \end{itemize}
\end{frame}

\begin{frame}{3.5 Considerações de extensibilidade}
    \begin{itemize}
        \item Checagem de erros
        \item Monitoramento/telemetria contínua
        \item Extensão do protocolo (novos comandos)
        \item Segurança (prevenção do uso indesejado)
        \item Uso em ensaios em vôo
    \end{itemize}
\end{frame}

\section{4. Testes e Resultados}
\begin{frame}{4.1 Metodologia e Testes}
    \begin{columns}
        \column{0.5\textwidth}
            \textbf{Hardware}
            \begin{itemize}
                \item Plataforma: Arduino UNO e "frame locking"
                \item Instrumentação de código ferramenta Host
                \item Instrumentação de código embarcado
                \item Osciloscópio FNIRSI DSO-153
            \end{itemize}
        \column{0.5\textwidth}
            \textbf{Cenários testados}
            \begin{itemize}
                \item Teste de Latência
                \item Teste de estresse
                \item Teste de Rajada (burst)
                \item Telemetria contínua a 10 Hz, 100 Hz e 1 kHz
            \end{itemize}
    \end{columns}
\end{frame}

\begin{frame}{4.2 Resultados – Teste de Latência}
    \centering
    \includegraphics[height=0.9\textheight]{recursos/figuras/teste_latencia_grafico.png}

\end{frame}

\begin{frame}{4.3 Resultados – Teste de estresse}
    \centering
    \includegraphics[height=0.9\textheight]{recursos/figuras/teste_estresse_grafico.png}
\end{frame}

\begin{frame}{4.4  Teste de Rajada (burst)}
    \centering
    \includegraphics[height=0.9\textheight]{recursos/figuras/teste_rajada_grafico.png}
\end{frame}

\begin{frame}{4.5 Telemetria contínua a 10 Hz, 100 Hz e 1 kHz}
\FloatBarrier
\begin{table}[htbp]
\centering
\caption{Síntese dos resultados dos testes com osciloscópio}
\label{tab:sintese_cenario2}
\begin{tabular}{|p{5cm}|p{2.5cm}|p{2.5cm}|p{2.5cm}|}
\hline
\textbf{Métrica} & \textbf{10 Hz} & \textbf{100 Hz} & \textbf{1 kHz} \\
\hline
\multicolumn{4}{|l|}{\textbf{Frequência de detecção dos pulsos de comando (PIN\_BUSY)}} \\
\hline 
Frequência Medida & 14.81 Hz & 99.13 Hz & 99.13 Hz \\
\hline
Período & 43.34 ms & 10.09 ms & 10.09 ms \\
\hline
Largura Pulso & 720 µs & 732 µs & 730 µs \\
\hline
Duty Cycle & 1.76\% & 7.26\% & 7.24\% \\
\hline
\multicolumn{4}{|l|}{\textbf{Medição de frame rate}} \\
\hline
Frequência & 49.55 Hz & 49.55 Hz & 49.70 Hz \\
\hline
Período Total & 20.18 ms & 20.18 ms & 20.12 ms \\
\hline
Período Loop & 10.09 ms & 10.09 ms & 10.06 ms \\
\hline
Duty Cycle & 50.02\% & 50.00\% & 50.00\% \\
\hline
Jitter perceptível no loop & Não observado & Não observado & Não observado \\
\hline
\end{tabular}
\end{table}

\end{frame}

\section{5. Conclusão}
\begin{frame}{5.1 Conclusão}
    \begin{block}{Objetivos atingidos}
        \vspace{0.5cm}
        \checkmark Protocolo leve especificado e implementado \\
        \checkmark Funciona em hardware real (Arduino UNO) \\
        \checkmark Overhead temporal quantificado e baixo até 100 Hz \\
        \checkmark Ferramenta host com protocolo funcional 
    \end{block}
\end{frame}

\begin{frame}{5.2 Limitações Observadas}
    \begin{itemize}
        \item Ausência de monitoramento/telemetria contínua
        \item Sem mecanismos de integridade
        \item Sem travas de segurança ou validação de dados
        \item Dependência de UART (taxa máxima 115200 bps)
    \end{itemize}
\end{frame}

\begin{frame}{5.3 Trabalhos Futuros}
    \begin{itemize}
        \item Portar para RTOS utilizando MCU com maior capacidade
        \item Adicionar monitramento/telemetria contínua
        \item Implementar CRC32 + ACK/NACK
        \item Introduzir uma camada de abstração de hardware
        \item Suporte a múltiplos alvos simultâneos (ferramenta host)
    \end{itemize}
\end{frame}

\begin{frame}{Fim}
    \centering
    \huge Obrigado \\
    \vspace{1cm}
    \LARGE Perguntas?
    \vfill
    \small
    Código-Fonte disponível em: \\
    \url{https://github.com/sfadiga/destra}
\end{frame}

\end{document}