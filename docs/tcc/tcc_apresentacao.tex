\documentclass{beamer}

% Tema recomendado
\usetheme{Metropolis}
\usecolortheme{default}

% Pacotes básicos
\usepackage[utf8]{inputenc}
\usepackage[brazil]{babel}
\usepackage{graphicx}
\usepackage{hyperref}
\usepackage{booktabs}

% Informações da apresentação
\title[DESTRA]{Depuração de Sistemas em Tempo Real: Uma Abordagem de Instrumentação de Código para Testes de Sistemas Críticos.}
\author{Sandro Fadiga}
\institute{Escola de Engenharia de São Carlos Universidade de São Paulo - USP}
\date{2025}

\begin{document}

% --------------------------------------------------------------
\begin{frame}
    \titlepage
\end{frame}

% --------------------------------------------------------------
\begin{frame}{Agenda}
    \tableofcontents
\end{frame}

% --------------------------------------------------------------
\section{Introdução}

\begin{frame}{Contexto do Problema}
    \begin{itemize}
        \item Crescente complexidade de sistemas embarcados aeronáuticos.
        \item Dificuldade de depuração em ambientes finais (Iron Bird, HIL).
        \item Limitações de técnicas invasivas (breakpoints, JTAG).
    \end{itemize}
\end{frame}

% --------------------------------------------------------------
\section{Objetivos}

\begin{frame}{Objetivo Geral}
    Desenvolver um protocolo de depuração não invasivo baseado em Peek/Poke
    para execução determinística de testes de integração em sistemas embarcados.
\end{frame}

\begin{frame}{Objetivos Específicos}
    \begin{itemize}
        \item Criar protocolo seguro e estruturado para comandos remotos.
        \item Implementar ferramenta host multiplataforma.
        \item Validar a solução em cenários reais (HIL-like).
        \item Garantir rastreabilidade e reprodutibilidade dos testes.
    \end{itemize}
\end{frame}

% --------------------------------------------------------------
\section{Fundamentação}

\begin{frame}{Debugging em Sistemas Embarcados}
    \begin{itemize}
        \item Técnicas tradicionais (JTAG, SWD) são invasivas.
        \item Falta de acesso interno em ambientes certificados.
        \item Peek/Poke como mecanismo universal de instrumentação.
    \end{itemize}
\end{frame}

% --------------------------------------------------------------
\section{Protocolo DESTRA}

\begin{frame}{Arquitetura da Solução}
    \begin{figure}
        \includegraphics[width=0.8\linewidth]{figuras/arquitetura.png}
    \end{figure}
\end{frame}

\begin{frame}{Formato dos Pacotes}
    \begin{itemize}
        \item Pacotes request/response
        \item Framing explícito
        \item CRC-16 para integridade
        \item Identificação de sequência
    \end{itemize}
\end{frame}

% --------------------------------------------------------------
\section{Resultados}

\begin{frame}{Testes de Estresse}
    \begin{itemize}
        \item Zero erros em 60 segundos contínuos.
        \item Frame rate estável.
        \item Nenhum gap de sequência.
    \end{itemize}
\end{frame}

\begin{frame}{Tabela Resumo}
    \begin{table}
        \centering
        \begin{tabular}{lll}
            \toprule
            Aspecto & Resultado \\
            \midrule
            Latência média & 6.63 ms \\
            Jitter médio & 0.23 ms \\
            Frames/s & 99 fps \\
            \bottomrule
        \end{tabular}
    \end{table}
\end{frame}

% --------------------------------------------------------------
\section{Conclusões}

\begin{frame}{Conclusões}
    \begin{itemize}
        \item Protocolo funciona de forma determinística.
        \item Reutilização de testes em múltiplos ambientes.
        \item Baixa intrusão → ideal para ambientes aeronáuticos.
    \end{itemize}
\end{frame}

\begin{frame}{Trabalhos Futuros}
    \begin{itemize}
        \item Adaptação para protocolos industriais (CAN, Ethernet).
        \item Versão com criptografia.
        \item Integração com testbeds HIL de alta fidelidade.
    \end{itemize}
\end{frame}

% --------------------------------------------------------------
\begin{frame}{Código-Fonte}
    Disponível em: \\
    \url{https://github.com/sfadiga/destra}
\end{frame}

% --------------------------------------------------------------
\begin{frame}
    \centering
    \Huge Obrigado!
\end{frame}

\end{document}
